	

\vspace{-4mm}

	
\subsection{Projections}
\label{vectors:projections}

\vspace{-2mm}

Dot products are used for computing projections.
Assume you're given two vectors $\vec{u}$ and $\vec{n}$ and you want to find the component
of $\vec{u}$ that points in the $\vec{n}$ direction.
The following formula based on the dot product will give you the answer:
\[
 \Pi_{\vec{n}}( \vec{u} ) \equiv \frac{  \vec{u} \cdot \vec{n}  }{ \| \vec{n} \|^2 } \vec{n}.
\]

\vspace{-2mm}

\noindent
This is how to implement this formula in \texttt{SymPy}:
\small
\begin{verbatimtab}
>>> u = Matrix([4,5,6])
>>> n = Matrix([1,1,1])
>>> (u.dot(n) / n.norm()**2)*n
[5, 5, 5]                      # projection of v in the n dir
\end{verbatimtab}
\normalsize

\noindent
In the case where the direction vector $\hat{n}$ is of unit length $\|\hat{n}\| = 1$,
the projection formula simplifies to $\Pi_{\hat{n}}( \vec{u} ) \equiv (\vec{u}\cdot\hat{n})\hat{n}$.


Consider now the plane $P$ defined by $(1,1,1)\cdot[(x,y,z)-(0,0,0)]=0$.
A plane is a two dimensional subspace of $\mathbb{R}^3$.
We can decompose any vector $\vec{u} \in \mathbb{R}^3$ into two parts $\vec{u}=\vec{v} + \vec{w}$
such that $\vec{v}$ lies inside the plane and $\vec{w}$ is perpendicular to the plane (parallel to $\vec{n}=(1,1,1)$).

To obtain the perpendicular-to-$P$ component of $\vec{u}$,
compute the projection of $\vec{u}$ in the direction $\vec{n}$:
\small
\begin{verbatimtab}
>>> w = (u.dot(n) / n.norm()**2)*n
[5, 5, 5]
\end{verbatimtab}
\normalsize

\noindent
To obtain the in-the-plane-$P$ component of $\vec{u}$,
start with $\vec{u}$ and subtract the perpendicular-to-$P$ part:
\small
\begin{verbatimtab}
>>> v = u - (u.dot(n)/n.norm()**2)*n      # same as u - w
[ -1, 0, 1]
\end{verbatimtab}
\normalsize

\noindent
You should check on your own that $\vec{v}+\vec{w}=\vec{u}$ as claimed.

\vspace{-5mm}
